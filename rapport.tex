%%%%%%%%%%%%%%%%%%%%%%%%%%%%%%%%%%%%%%%%%%%%%%%%%%%%%%%%%%%
%%%%% 													%%%%%
%%%%%	Aller voir le lien suivant :					%%%%%
%%%%%													%%%%%
%%%%%	http://fr.wikibooks.org/wiki/LaTeX				%%%%%
%%%%%													%%%%%
%%%%%%%%%%%%%%%%%%%%%%%%%%%%%%%%%%%%%%%%%%%%%%%%%%%%%%%%%%%

\documentclass[a4paper,12pt]{report}

%%%%%%%%%%%%%%%%%%%%%%%%%%%%%%%%%%%%%%%%%%%%%%%%%%%%%%%%%%%
%%%%% 	pour le français et les accents 	      %%%%%
%%%%%%%%%%%%%%%%%%%%%%%%%%%%%%%%%%%%%%%%%%%%%%%%%%%%%%%%%%%

\usepackage[utf8]{inputenc} 
\usepackage[french]{babel} 

%%%%% 		Pour les marges de la page			 %%%%%%%%
\usepackage[top=3cm, bottom=3cm, left=3cm, right=3cm]{geometry}
%%%%%%%%%%%%%%%%%%%%%%%%%%%%%%%%%%%%%%%%%%%%%%%%%%%%%%%%%%%


\usepackage[T1]{fontenc}
\usepackage{lmodern}

%%%%% Pour les url %%%%%
\usepackage{hyperref}

%%%%% Gestion des numeros de sous-sous-sections %%%%%
\usepackage{titlesec}
\setcounter{secnumdepth}{3}

%%%%% Gestion des sous sous section dans la TOC %%%%%
\setcounter{tocdepth}{3}

%%%%% Insertion des images %%%%%
\usepackage{graphicx}

%%%%% Gestion du centrage des titres %%%%%
\usepackage{sectsty}
\usepackage{lipsum}

%%%%% Demarrage à 0 des sections %%%%%%
\renewcommand{\thesection}{\arabic{section}}
\setcounter{section}{0}


%%% Maketitle metadata
\newcommand{\horrule}[1]{\rule{\linewidth}{#1}} 	%ligne horizontale

\parindent=2cm

%%% Forcer les images à rester dans leur section
\usepackage[section]{placeins}

%%% Utiliser un formatage pour le code 
\usepackage{listings}
\usepackage{color}
\definecolor{lightgray}{rgb}{.9,.9,.9}
\definecolor{darkgray}{rgb}{.4,.4,.4}
\definecolor{purple}{rgb}{0.65, 0.12, 0.82}

\lstdefinelanguage{JavaScript}{
  keywords={typeof, new, true, false, catch, function, return, null, catch, switch, var, if, in, while, do, else, case, break},
  keywordstyle=\color{blue}\bfseries,
  ndkeywords={class, export, boolean, throw, implements, import, this},
  ndkeywordstyle=\color{darkgray}\bfseries,
  identifierstyle=\color{black},
  sensitive=false,
  comment=[l]{//},
  morecomment=[s]{/*}{*/},
  commentstyle=\color{purple}\ttfamily,
  stringstyle=\color{red}\ttfamily,
  morestring=[b]',
  morestring=[b]"
}

\lstset{
   language=JavaScript,
   backgroundcolor=\color{lightgray},
   extendedchars=true,
   basicstyle=\footnotesize\ttfamily,
   showstringspaces=false,
   showspaces=false,
   numbers=left,
   numberstyle=\footnotesize,
   numbersep=9pt,
   tabsize=2,
   breaklines=true,
   showtabs=false,
   captionpos=b
}


%%%%%%%%%%%%%%%%%%%  DEBUT DU RAPPORT %%%%%%%%%%%%%%%%%%%%
\begin{document}
\sffamily


%%%%%%%%%%%%%%%%%%%  PAGE DE GARDE %%%%%%%%%%%%%%%%%%%%%%
\begin{titlepage}
  \begin{sffamily}
  \begin{center}

    % Upper part of the page. The '~' is needed because \\
    % only works if a paragraph has started.
    \includegraphics[scale=0.5]{lille1.png}~\\[2cm]

    \textsc{\LARGE MASTER 2 TIIR}\\[0.2cm]

    \textsc{\large Rapport de stage de fin d'études}\\[4cm]

    % Title
    \horrule{0.5pt} \\[0.4cm]
    { \huge \bfseries Analyste d'exploitation junior\\[0.4cm] }
    \horrule{0.5pt} \\[2.5cm]
    \includegraphics[scale=0.5]{sogeti.png}
    \\[3cm]

    % Author and supervisor
    \begin{minipage}{0.4\textwidth}
      \begin{flushleft} \large
        \emph{Auteur :}\\
        Kevin Franquenouille\\
      \end{flushleft}
    \end{minipage}
    \begin{minipage}{0.4\textwidth}
      \begin{flushright} \large
        \emph{Encadrants :}\\
        Samy Meftali\\
        Ludovic Gidel
      \end{flushright}
    \end{minipage}

    \vfill

    % Bottom of the page
    {\large 1\ier{} Mars 2016 — 31 Août 2016}

  \end{center}
  \end{sffamily}
\end{titlepage}
%\maketitle	

%%%%%%%%%%%%%%%%%%%  DEBUT REMERCIEMENTS %%%%%%%%%%%%%%%%%%%%
\newpage
\begin{center}
\section*{Remerciements}
\end{center}
\paragraph{}
Avant toutes choses, je tenais à remercier l'ensemble des collaborateurs de Sogeti Nord m'avoir accueilli au sein de leur entreprise et pour m'avoir accordé leur confiance et également de m'avoir proposé un CDI.

\paragraph{}
Je tenais aussi à remercier Ludovic Gidel ainsi que les équipes du CDSI (Centre de Service Intégration) et du CDSP (Centre de Service Production) pour m'avoir aidé et suivi dans les différentes missions qui m'ont été confiées.

\paragraph{}
Je souhaite égalemennt remercier Samy Meftali qui a été mon tuteur universitaire et qui m'a accompagné pendant ce stage Ce dernier m'a permis également de terminer plus tôt mon stage afin de commencer au plus vite mon CDI et d'être placé en clientèle plus rapidement.

%%%%%%%%%%%%%%%%%%%  FIN REMERCIEMENTS %%%%%%%%%%%%%%%%%%%%

%%%%%%%%%%%%%%%%%%%  DEBUT INTRODUCTION %%%%%%%%%%%%%%%%%%%%
\newpage
\begin{center}
\section*{Introduction}
\end{center}
\paragraph{}
Pour la dernière année de Master TIIR, nous avons eu la possibilité de réaliser un stage de fin allant de 4 à 6 mois. J'ai personnellement choisis de prendre un stage de 6 mois afin d'acquérir le plus d'expérience possible.\\
Il s'agit également d'un stage de pré-embauche qui s'est terminé par un CDI pour mon cas.

\paragraph{}
L'objectif de mon stage est de découvrir la partie infrastructure avec les différentes missions que l'on peut accomplir et également voir quelques technologies utilisées pour réaliser cela.\\
Mon rôle consistait à appuyer l'équipe du CDSI, en particulier la partie Unix, afin de pouvoir apporter mes connaissances Java.\\
J'ai également apporté mon aide et mes connaissances au CDSP afin de réaliser des scripts et un migration de machine.

\paragraph{}
Dans un premier temps, je présenterai la société Sogeti, l'entreprise dans laquelle j'ai effectué mon stage, ses différentes activités, ainsi que les objectifs de mon stage.

\paragraph{}
Dans un second temps, je détaillerai les activités réalisés sur la partie du CDSI avec les différents outils utilisés tel que XFB Gateway, Absyss VTOM (Visual Time Operation Manager) ainsi que Oracle WebLogic Server.

\paragraph{}
Dans un dernier temps, je préciserai quelques missions du CDSP sur lesquelles j'ai pu apporté mes compétences notamment sur la partie scripting et également pour un projet de migration de shell VTOM.

\paragraph{}
Enfin, je terminerai ce rapport par un bilan de ces six mois de stage en entreprise, ainsi que par une présentation des bénéfices du stage, pour moi d'une part, et pour l'entreprise
d'autre part.


%%%%%%%%%%%%%%%%%%%  FIN INTRODUCTION %%%%%%%%%%%%%%%%%%%%

%%%%%%%%%%%%%%%%%%%  DEBUT TABLE DES MATIERES %%%%%%%%%%%%%%%%%%%%
\tableofcontents
%%%%%%%%%%%%%%%%%%%  FIN TABLE DES MATIERES %%%%%%%%%%%%%%%%%%%%

%%%%%%%%%%%%%%%%%%%  DEBUT %%%%%%%%%%%%%%%%%%%%
\newpage
\section{Sogeti}
	\subsection{Hiérarchie}
	\paragraph{}
	Capgemini-Sogeti est un groupe international de grande avec de
nombreuses branches dans chacun des pays en font de lui une structure complexe qui reste très organisée. Voici comment Sogeti France est organisé pour arriver vers l'entité au sein de laquelle j'évolue: Sogeti Infrastructure.
	\paragraph{}
	\includegraphics[scale=0.7]{architecture.png}

	\paragraph{}
	La partie infrastructure de Sogeti est divisée en 3 parties :
	\begin{itemize}
		\item[•] Grand Est (Nord \& Est, Rhône-Alpes, Méditerranée)
		\item[•] Grand Ouest (Midi-Pyrénées, Aquitaine, Ouest)
		\item[•] Ile de France
	\end{itemize}	
	\paragraph{}
	Pour chaque partie, il y a un Business Line Manager (BLM) ainsi qu'un ou plusieurs Practice Manager (PM).\\
	Dans le cas de la BU (Business Unit) de Villeneuve d'Ascq, elle est rattachée à l'entité \textit{Nord \& Est} avec comme BLM, Jérôme Galland. Les Practice Manager côté Infrastructure sont Christian Mascaux et Hugues Allart.

	\subsection{Activités}
	\paragraph*{}
	Sogeti est une filiale du groupe Capgemini avec environ 20 000 collaborateurs présents sur 15 pays en Europe, en Inde et dans les Etats-Unis. Parmi ces collaborateurs, 6 000 collaborateurs sont répartis sur 21 sites français. C'est également l’un des leaders des services technologiques et du test logiciel, spécialisé dans la gestion des applicatifs, des infrastructures et les services en ingénierie.\\
	Sogeti se décompose en 4 domaines d'interventions :\\
	\begin{itemize}
		\item[•] Application
		\item[•] Testing
		\item[•] Sécurité
		\item[•] Infrastructure
	\end{itemize}
	
	\subsection{Présentation de l'équipe}
	\paragraph*{}
	Pour le client CACF (Crédit Agricole Consumer Finance), 2 équipes sont présentes :\\
	\begin{itemize}
		\item[•] Le CDSI (Centre De Service Intégration)
		\item[•] Le CDSP (Centre De Service Production)
	\end{itemize}
	
	\paragraph*{}
	Le CDSP est une équipe composée de 3 personnes, Nicolas France, chef de projet junior, Pierre Wrobel, chef de projet junior également et Ludovic Gidel chef de projet mais aussi responsable du client CACF.
	
	\paragraph*{}
	Le CDSI est séparé en 2 mondes : le monde Unix et le monde MVS.\\
	J'ai l'opportunité de travailler sur le côté Unix avec Laurent Mattu sur les technologies Oracle WebLogic Server, VTOM ainsi que XFB Gateway. Nous travaillons en collaboration, avec le côté MVS pour tout ce qui est gestion des flux avec MVS et VTOM.	Le travail est uniquement orienté sur la partie \textit{Recette} des applications
	
	\paragraph{}
	Le CDSI et le CDSP travaillent aussi conjointement pour certaines demandes, en particulier le cas de mise en recette de certaines applications.
	
\newpage	
\section{Activités du CDSI (Centre De Service Intégration)}
	\subsection{Côté Mainframe (MVS)}
	\paragraph*{}
	JCL, Flux, MVS reste cependant bien adapté au traitement par lots (batch), pris en charge par le langage JCL.
	
	\subsection{Côté Open (Unix)}
	\paragraph*{}
	Weblogic, VTOM, XFB Gateway 
	
\newpage	
\section{Activités de CDSP (Centre De Service Production)}
	\subsection{Migration de batch VTOM}
	\paragraph*{}
	
	
	\subsection{Activités annexes}
	\paragraph*{}
	
	
\newpage	
\begin{center}
\section*{Conclusion}
\end{center}
\paragraph*{}



\end{document}
